\documentclass[addpoints,12pt]{exam}
  \usepackage[a4paper]{geometry}
  \usepackage{enumitem}
  \usepackage{amsmath,stackengine}
  \geometry{
 	a4paper,
 	total={150mm,257mm},
 	left=25mm,
 	top=20mm,
  }
  
  \begin{document}
  
  \hspace{-7mm}ID No.\rule{20mm}{0.3mm}
  
  \begin{center} 
	%college Heading
  
  \textbf{Birla Vishwakarma Mahavidhyalaya(Engineering College)} \\
  \textbf{\textit{(An Autonomous Institute)}} \\
  \textbf{First Year, B.Tech} \\
  \textbf{1st Mid Semester Examination ,Odd,AY AY 2022-23} \\
  \vspace{4mm}
  
  
  \end{center}
  
	%Course code, title, maximum marks, date, time
  \hspace{-7mm}
  \parbox[t]{50mm}{\textbf{Course Code: 3CP02}}
  \parbox[t]{100mm}{\textbf{Course Title: General Knowledge}}\vspace{2mm}\\
  \parbox[t]{50mm}{\textbf{Date: 20.09.21}}
  \parbox[t]{75mm}{\textbf{Time : 00:00 AM to 01:00 AM}}
  \parbox[t]{50mm}{\textbf{Maximum Marks: 30}}\\
  \hline \vspace{2mm} 
  \hspace{-6mm}\textbf{Instruction}

  
	%instruction section
	
  \begin{itemize}[leftmargin=4mm,rightmargin=-2cm]
      \item Numbers in the square brackets to the right indicate maximum marks.
      \item Instruction Demo
      \item The text just below marks indicates the Course Outcome Nos. (CO) followed by the Bloom’s taxonomy level of the question, i.e., R: Remember, U: Understand, A: Apply, N: Analyze, E: Evaluate, C: Create
  \end{itemize}
  \hline
  \vspace{5mm}
	%Questions section  
  
  \begin{questions}
\pointname{}
\pointsinrightmargin
\pointformat{\parbox[t]{16pt}{\text{[\thepoints]}}}
\question[20]
\vspace{-\baselineskip}\vspace{1.5mm}Answer The Following Question
\begin{parts}
\part what is nfa?
\part what is nfa explain?
\part why is dfa used?
\end{parts}

\end{questions}

\end{document}