\documentclass[addpoints,12pt]{exam}
  \usepackage[a4paper]{geometry}
  \usepackage{enumitem}
  \usepackage{amsmath,stackengine}
  \geometry{
  a4paper,
  total={150mm,257mm},
  left=25mm,
  top=20mm,
  } 
    

 
  \begin{document}
    \hspace{-7mm}ID No.\rule{20mm}{0.3mm}
    \begin{center}
 \textbf{Birla Vishwakarma Mahavidhyalaya(Engineering College)} \\
  \textbf{\textit{(An Autonomous Institute)}} \\
  \textbf{First Year, B.Tech} \\
  \textbf{1st Mid Semester Examination ,Odd,AY AY 2022-23} \\
  \vspace{4mm}
 
 
  \end{center}
 
%Course code, title, maximum marks, date, time
  \hspace{-7mm}
  \parbox[t]{50mm}{\textbf{Course Code: SCP01}}
  \parbox[t]{100mm}{\textbf{Course Title: sample course}}\vspace{2mm}\\
  \parbox[t]{50mm}{\textbf{Date: 2022.10.07}}
  \parbox[t]{75mm}{\textbf{Time : 00:02 AM to 00:03 AM}}
  \parbox[t]{50mm}{\textbf{Maximum Marks: 30}}\\
  \line(1,0){170mm} \vspace{2mm}
  \hspace{-6mm}\textbf{Instruction}

 
%instruction section

  \begin{itemize}[leftmargin=4mm,rightmargin=-2cm]
      \item Numbers in the square brackets to the right indicate maximum marks.
     
      \item The text just below marks indicates the Course Outcome Nos. (CO) followed by the Bloom’s taxonomy level of the question, i.e., R: Remember, U: Understand, A: Apply, N: Analyze, E: Evaluate, C: Create
  \end{itemize}
  \line(1,0){170mm}
 \vspace{5mm}
\begin{questions}
\pointname{}
\pointsinrightmargin
\pointformat{\parbox[t]{16pt}{\text{[\thepoints]}\\ 1,2A}}\question[2]
what is sample in course ?
\end{questions}
\end{document}