\documentclass[addpoints,12pt]{exam}
  \usepackage[a4paper]{geometry}
  \usepackage{enumitem}
  \usepackage{amsmath,stackengine}
  \geometry{
  a4paper,
  total={150mm,257mm},
  left=25mm,
  top=20mm,
  } 
    

 
  \begin{document}
    \hspace{-7mm}ID No.\rule{20mm}{0.3mm}
    \begin{center}
 \textbf{Birla Vishwakarma Mahavidhyalaya(Engineering College)} \\
  \textbf{\textit{(An Autonomous Institute)}} \\
  \textbf{First Year, B.Tech} \\
  \textbf{1st Mid Semester Examination ,Odd,AY AY 2022-23} \\
  \vspace{4mm}
 
 
  \end{center}
 
%Course code, title, maximum marks, date, time
  \hspace{-7mm}
  \parbox[t]{50mm}{\textbf{Course Code: 3cp02}}
  \parbox[t]{100mm}{\textbf{Course Title: General Knowledge}}\vspace{2mm}\\
  \parbox[t]{50mm}{\textbf{Date: 2022.09.21}}
  \parbox[t]{75mm}{\textbf{Time : 00:00 AM to 00:00 AM}}
  \parbox[t]{50mm}{\textbf{Maximum Marks: 30}}\\
  \line(1,0){170mm} \vspace{2mm}
  \hspace{-6mm}\textbf{Instruction}

 
%instruction section

  \begin{itemize}[leftmargin=4mm,rightmargin=-2cm]
      \item Numbers in the square brackets to the right indicate maximum marks.
     \item You will get UFM if Write Paper,\item Person named nikunj will fail for sure
      \item The text just below marks indicates the Course Outcome Nos. (CO) followed by the Bloom’s taxonomy level of the question, i.e., R: Remember, U: Understand, A: Apply, N: Analyze, E: Evaluate, C: Create
  \end{itemize}
  \line(1,0){170mm}
 \vspace{5mm}
\begin{questions}
\pointname{}
\pointsinrightmargin
\pointformat{\parbox[t]{16pt}{\text{[\thepoints]}}}
\question[10]
\vspace{3mm}Answer Jobless Questions.
\begin{parts}
\part Mr. Thomas invested an amount of Rs. 13,900 divided in two different schemes A and B at the simple interest rate of 14% p.a. and 11% p.a. respectively. If the total amount of simple interest earned in 2 years be Rs. 3508, what was the amount invested in Scheme B?
\part Three pipes A, B and C can fill a tank from empty to full in 30 minutes, 20 minutes, and 10 minutes respectively. When the tank is empty, all the three pipes are opened. A, B and C discharge chemical solutions P,Q and R respectively. What is the proportion of the solution R in the liquid in the tank after 3 minutes?
\end{parts}
\question[15]
\vspace{3mm}Answer Some Easy Questions.
\begin{parts}
\part A bank offers 5% compound interest calculated on half-yearly basis. A customer deposits Rs. 1600 each on 1st January and 1st July of a year. At the end of the year, the amount he would have gained by way of interest is:
\part A, B, C rent a pasture. A puts 10 oxen for 7 months, B puts 12 oxen for 5 months and C puts 15 oxen for 3 months for grazing. If the rent of the pasture is Rs. 175, how much must C pay as his share of rent?
\part what is your name in real life?
\end{parts}
\question[20]
\vspace{3mm}Answer Some Paplu Questions
\begin{parts}
\part what is nfa?
\part what is nfa explain?
\part what is money?
\part what is lucky draw?
\end{parts}
\question[5]
\vspace{3mm}Two pipes A and B together can fill a cistern in 4 hours. Had they been opened separately, then B would have taken 6 hours more than A to fill the cistern. How much time will be taken by A to fill the cistern separately?
\end{questions}
\end{document}